% resumo em inglês
\begin{resumo}[Abstract]
 \begin{otherlanguage*}{english}
   
  This work aims to explore the application of artificial neural networks (ANNs) and deep reinforcement learning (DRL) in the creation of intelligent agents in electronic games, using machine learning techniques to develop agents that interact dynamically with the environment, players and other agents, adapting their behaviors based on acquired and accumulated experiences. The methodology involves implementing an artificial neural network that processes complex inputs from the environment and adjusts the behavior of agents through the use of deep reinforcement learning. The ReLU activation function was chosen to ensure efficiency in training the neural networks. The agents' performance will be evaluated through their adaptability, the uniqueness of their behaviors and their coherence with the virtual environment in which they find themselves, with continuous adjustments to improve these parameters.

   \vspace{\onelineskip}
 
   \noindent 
   \textbf{Key-words}: artificial neural networks, deep reinforcement learning, intelligent agents, game development.
 \end{otherlanguage*}
\end{resumo}