\chapter[Revisão de Literatura]{Revisão de Literatura}\label{capitulo2}
\addcontentsline{toc}{chapter}{Revisão de Literatura}

A inteligência artificial é uma parte inseparável da maioria dos jogos eletrônicos existentes, pois muitos deles possuem adversários ou auxiliares que definem parte integral da experiência do jogo, e é amplamente utilizada para criar comportamentos interessantes para os agentes inteligentes, contribuindo para a imersão e complexidade dos desafios oferecidos aos jogadores \cite{dill2015whatis}.

Diversas aplicações de IA, incluindo redes neurais artificiais, já são usadas em jogos de diferentes tipos. No entanto, muitos jogos não disponibilizam seu código-fonte publicamente, nem os métodos utilizados para alcançar os resultados vistos. Essa falta de transparência limita a análise e compreensão dos métodos de IA aplicados em diferentes contextos de jogos eletrônicos.

O estudo "\textit{Player-IA Interaction: What Neural Network Games Reveal About AI as Play}", de Zhu et al. (2021), apresenta uma coletânea de jogos onde redes neurais são utilizadas de diferentes formas e discute a interação humano-máquina. O estudo classifica as redes neurais em quatro tipos principais: aprendizes (a rede aprende com o jogador a executar uma tarefa), competidores (a rede aprende como o jogador joga para gerar desafios), projetistas (a rede neural cria elementos para interação do jogador) e parceiros (a rede e o jogador colaboram em direção a um objetivo comum). Esse estudo é particularmente relevante para dar a capacidade aos agentes de se encaixarem dinamicamente nas funções descritas em tempo real através da interação com o jogador, outros agentes, ou o ambiente se ajustando a diferentes papéis conforme o contexto.

O livro "\textit{Game AI Pro}", editado por Rabin et al. (2015), aborda uma vasta gama de tópicos relacionados à IA em jogos, desde a base teórica do funcionamento dos neurônios biológicos até aplicações práticas e complexas, como o uso de redes neurais para criar experiências imersivas. O livro é dividido em capítulos, cada um escrito por diferentes profissionais da indústria de jogos e pesquisadores da área de IA, oferecendo uma visão diversificada e abrangente sobre o impacto da IA em jogos. Tanto os tópicos teóricos quanto os tópicos mais abstratos são de extrema relevância para o projeto atual, já que as aplicações dessas técnicas de IA devem oferecer ao jogador uma experiência singular, mas ainda assim agradável e imersiva.

O livro "\textit{Artificial Intelligence: A Modern Approach}", de Russell e Norvig (2020), é uma das obras mais abrangentes sobre IA. Foca nos fundamentos da IA, agentes inteligentes e sua interação com o ambiente, resolução de problemas e lógica. Além disso, o livro aborda a aplicação de IA em jogos estratégicos, como o xadrez, no contexto de algoritmos de busca e raciocínio estratégico, fornecendo uma base sólida para a compreensão dos conceitos fundamentais da IA aplicados a jogos. Esses conceitos fornecem a base teórica necessária para desenvolver algoritmos que permitam aos agentes aprender e se adaptar a partir de suas interações.

O livro "\textit{Neural Networks and Deep Learning: A Textbook}", de Aggarwal (2018), explora o uso de redes neurais profundas para aproximar funções de valor e funções de ação-valor em aprendizado por reforço profundo. O livro oferece uma análise detalhada sobre como as redes neurais podem ser empregadas para aprender e otimizar políticas de decisão em ambientes complexos. Isso é crucial para o objetivo de implementar um modelo de inteligência artificial que dê individualidade aos agentes no jogo através de suas diferentes experiências “vividas”, ao permitir que cada agente aprenda e se adapte de forma única, criando comportamentos emergentes que refletem suas interações e experiências passadas.

O artigo "\textit{Human-level Control Through Deep Reinforcement Learning}", de Mnih et al. (2015), demonstra a aplicação de redes neurais profundas para alcançar controle em nível humano em jogos, como Atari. O estudo é fundamental para compreender a eficácia das técnicas de aprendizado por reforço profundo em jogos complexos, permitindo que agentes aprendam e otimizem estratégias avançadas, se correlacionando com o objetivo de dar a capacidade aos agentes de aprenderem e modificarem seus comportamentos em tempo real.

O artigo "\textit{Continuous Control With Deep Reinforcement Learning}", de Lillicrap et al. (2016), explora o uso de DRL para controlar sistemas contínuos, como robôs. O estudo detalha como o DRL pode ser aplicado para realizar tarefas complexas em ambientes variados, incluindo manipulação de objetos e navegação, ajudando com o fornecimento de bases técnicas para o desenvolvimento de agentes que gerenciem o próprio comportamento.

O livro "\textit{Neural Networks and Learning Machines}", de Haykin (2008), fornece uma visão abrangente sobre redes neurais e suas aplicações, discutindo funções de ativação e métodos de treinamento. O artigo "\textit{Learning Representations by Back-Propagating Errors}", de Rumelhart, Hinton e Williams (1986), descreve o algoritmo de retropropagação, uma técnica fundamental para o treinamento de redes neurais. Ambos os estudos são cruciais para formar uma base técnica para o desenvolvimento das redes neurais artificiais.

O livro "\textit{Neural Networks and Deep Learning: A Textbook}", de Nielsen (2015), discute o treinamento de redes neurais e a detecção de overfitting, oferecendo detalhes sobre como garantir que a rede neural generalize bem e não apenas memorize os dados de treinamento. Isso é essencial para a implementação de um modelo de IA que dê individualidade aos agentes no jogo, assegurando que o comportamento dos agentes seja único e dinâmico.