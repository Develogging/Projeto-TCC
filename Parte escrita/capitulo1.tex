
% Introdução 
\chapter[Introdução]{Introdução}\label{capitulo1}
\addcontentsline{toc}{chapter}{Introdução}

A inteligência artificial (IA) é parte integral do desenvolvimento de jogos, onde sua aplicação impacta diretamente a experiência do jogador. Esse campo de estudo está em constante evolução, e suas décadas de conhecimento acumulado são usadas das mais diversas maneiras para a criação de variados agentes inteligentes em jogos eletrônicos \cite{dill2015whatis}.

Uma das formas mais sofisticadas de inteligência artificial existentes atualmente são as redes neurais artificiais, que podem corroborar para a criação de IAs mais inteligentes e com a capacidade de tomar decisões mais complexas, através da captação de dados do ambiente virtual em que se encontra, o processamento desses dados pelo modelo de rede aplicado e a determinação de ações a serem tomadas em sua saída \cite{nielsen2015neural}.

O modelo de aprendizado por reforço profundo é especialmente interessante como estrutura de tomada de decisões para agentes inteligentes em jogos eletrônicos por sua capacidade de aprendizado e adaptação dinâmicas, sua consideração sequencial de tomadas de decisão ao longo do tempo, considerando as consequências a longo prazo, e sua boa performance em ambientes altamente mutáveis e complexos \cite{mnih2015human}.

\section{Objetivos}

Este trabalho tem como objetivo o estudo e aplicação de redes neurais e aprendizado por reforço profundo na criação de agentes inteligentes em um jogo eletrônico que, ao decorrer de diferentes interações com o “mundo” em que se encontram, seja essa interação com o jogador, elementos estáticos do “mundo”, ou outros agentes inteligentes, modifique seu comportamento de forma a se adaptar às novas condições.

\subsection{Objetivo Geral}

Aprofundamento no estudo de redes neurais artificiais (RNAs) e aprendizado por reforço profundo (\textit{deep reinforcement learning}, ou DRL) com foco em sua aplicação na criação de inteligência artificial para agentes inteligentes em jogos eletrônicos, criando agentes mais adaptáveis ao ambiente atual e à interferência de terceiros.

\subsection{Objetivos Específicos}
\begin{itemize}
    \item Implementar um modelo de inteligência artificial que gerencie o comportamento de agentes inteligentes em um jogo eletrônico.
    \item Aplicar técnicas de aprendizado por reforço profundo para dar a capacidade aos agentes de aprenderem e modificarem seus comportamentos em tempo real através da interação com o jogador, outros agentes, ou o ambiente.
    \item Implementar um modelo de inteligência artificial que dê individualidade aos agentes no jogo através de suas diferentes experiências “vividas”.
    \item Comparar a aplicação da abordagem desenvolvida com abordagens já presentes no mercado e sua eficácia em criar agentes inteligentes e altamente adaptáveis.
    \item Gerar uma maior imersão do jogador no ambiente virtual através da realização desses agentes mais inteligentes e adaptáveis.
\end{itemize}