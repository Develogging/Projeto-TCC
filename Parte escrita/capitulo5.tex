\chapter[Resultados Esperados]{Resultados Esperados}\label{capitulo5}
\addcontentsline{toc}{chapter}{Resultados Esperados}

Ao fim do projeto, espera-se que a rede neural artificial esteja funcionando devidamente, e o aprendizado por reforço profundo esteja produzindo cenários interessantes criados através da experiência de cada agente.

Os agentes devem demonstrar inteligência superior se comparados a agentes que utilizam técnicas mais simples, como máquinas de estado finito. Também devem ser altamente adaptáveis, mesmo que ocorram mudanças bruscas no ambiente e em suas alterações.

Mesmo que os agentes tenham completa autonomia para modificar suas políticas de comportamento, ainda assim deve haver espaço para o design intencional de comportamento durante o desenvolvimento, a fim de criar cenários e interações mais interessantes.

Também é imprescindível que haja uma evolução do conhecimento sobre o desenvolvimento de jogos, redes neurais e aprendizado de máquina ao fim do projeto.

Por fim, o resultado mais importante é o de que o jogo esteja em um estado jogável, e que apresente engajamento e imersão suficientemente grandes, e que os comportamentos não se afunilem em certos padrões depois de certo tempo.