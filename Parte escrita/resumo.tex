% resumo em português
\setlength{\absparsep}{18pt} % ajusta o espaçamento dos parágrafos do resumo
\begin{resumo}
    
    Este trabalho tem como objetivo explorar a aplicação de redes neurais artificiais (RNAs) e aprendizado por reforço profundo (DRL) na criação de agentes inteligentes em jogos eletrônicos, utilizando técnicas de aprendizado de máquina para desenvolver agentes que interajam de maneira dinâmica com o ambiente, jogadores e outros agentes, adaptando seus comportamentos com base nas experiências adquiridas e acumuladas. A metodologia envolve a implementação de uma rede neural artificial que processa entradas complexas do ambiente e ajusta o comportamento dos agentes através do uso de aprendizado por reforço profundo. A função de ativação ReLU foi escolhida para garantir eficiência no treinamento das redes neurais. O desempenho dos agentes será avaliado através de sua adaptabilidade, a singularidade de seus comportamentos e a coerência dos mesmos para com o ambiente virtual em que se encontram, com ajustes contínuos para melhorar esses parâmetros.

 \textbf{Palavras-chaves}: redes neurais artificiais, aprendizado por reforço profundo, agentes inteligentes, desenvolvimento de jogos.
\end{resumo}